\section{Protocol summary}

% TODO
Describe stdin, stdout, json and shit...

\subsection{Types}

% TODO Speak about types

Provided types are semantic and do not define their underlying storage. Enums values are specified like so \texttt{enum \{ value-1, ..., value-n \}} and are to be provided in commands as strings.

\subsection{Command summary}

All commands contain at least a \texttt{command} field.

\begin{center}
\begin{tabu} to\linewidth {X[l, -2] | X[l, 4] | X[l, -1] | X[c, -1]}
    Command
        & Description
        & Parameters
        & Read-only \\ \hline \hline
    \texttt{reset}
        & Resets all internal state.
        &
        & \faClose{} \\
    \texttt{unserialize}
        & Initializes the application state with a saved game.
        & format, value
        & \faClose{} \\
    \texttt{serialize}
        & Serializes the game state.
        & format
        & \faCheck{} \\
    \texttt{generate\_moves}
        & Generates all moves for a board piece.
        & x, y
        & \faCheck{} \\
    \texttt{check\_move}
        & Checks if a move is valid.
        & from\_x, from\_y, to\_x, to\_y
        & \faCheck{} \\
    \texttt{move}
        & Moves a piece on the board updating internal states.
        & from\_x, from\_y, to\_x, to\_y
        & \faClose{} \\
    \texttt{exit}
        & Exits the program.
        &
        & \faClose{} \\
\end{tabu}
\end{center}

\subsection{Errors}

All responses contain at least a \texttt{status} field with either \texttt{ok} or \texttt{nok}. In case the \texttt{status} is \texttt{nok}, the response always contains an \texttt{error} field containing and error code (string) and an \texttt{error\_msg} field containing more information about the error.\\
Aside from command specific errors, some errors should always be handled when reading a \texttt{nok} response:

\begin{ResponseErrors}
    \texttt{invalid\_json}
        & Could not parse the provided json. \\
    \texttt{invalid\_command}
        & Invalid command provided. \\
    \texttt{invalid\_type}
        & The value of a field is invalid (invalid data type, value outside of enum bounds; etc.). \\
    \texttt{missing\_field}
        & A mandatory type was missing for the command. \\
\end{ResponseErrors}

\newpage
\section{Protocol commands}

\subsection{reset}

\subsubsection{Command parameters}

n/a

\subsubsection{Response parameters}

n/a

\subsubsection{Response errors}

n/a

\newpage
\subsection{unserialize}

\subsubsection{Command parameters}

\begin{CommandParameters}
    \texttt{format}
        & \texttt{enum \{ PGN, binary-64 \}}
        & Either \texttt{PGN} or \texttt{binary-64} (\hyperref[serialization]{§ \ref{serialization}}). \\
    \texttt{value}
        & \texttt{string}
        & The value to deserialize using the specified format. \\
    \texttt{lenient}
        & \texttt{boolean?}
        & If \texttt{true}, skips checking moves are valid (default: \texttt{false}). \\
\end{CommandParameters}

\subsubsection{Response parameters}

n/a

\subsubsection{Response errors}

\begin{ResponseErrors}
    \texttt{invalid\_format}
        & Provided value could not be parsed. \\
    \texttt{invalid\_data}
        & Input data is incorrect. \\
    \texttt{invalid\_move}
        & Input data contains an invalid move. \\
\end{ResponseErrors}

\newpage
\subsection{serialize}

\subsubsection{Command parameters}

\begin{CommandParameters}
    \texttt{format}
        & \texttt{enum \{ PGN, binary-64 \}}
        & Either \texttt{PGN} or \texttt{binary-64} (\hyperref[serialization]{§ \ref{serialization}}). \\
\end{CommandParameters}

\subsubsection{Response parameters}

\begin{ResponseParameters}
    \texttt{value}
        & \texttt{string}
        & The current game state serialized using the specified format. \\
\end{ResponseParameters}

\subsubsection{Response errors}

n/a

\newpage
\subsection{generate\_moves}

\subsubsection{Command parameters}

\begin{CommandParameters}
    \texttt{x}
        & \texttt{integer}
        & The column of the piece to generate moves for. \\
    \texttt{y}
        & \texttt{integer}
        & The row of the piece to generate moves for. \\
\end{CommandParameters}

\subsubsection{Response parameters}

\begin{ResponseParameters}
    \texttt{moves}
        & \texttt{[\,\{ x: integer,\linebreak y: integer \}]\,}
        & The list of moves available for the piece at the provided coordinates. \\
\end{ResponseParameters}

\subsubsection{Response errors}

\begin{ResponseErrors}
    \texttt{invalid\_coordinates}
        & Provided coordinates are outside the board. \\
    \texttt{no\_piece\_at\_coordinates}
        & There is no piece on the board at the provided coordinates. \\
\end{ResponseErrors}

\newpage
\subsection{check\_move}

\subsubsection{Command parameters}

\begin{CommandParameters}
    \texttt{from\_x}
        & \texttt{integer}
        & The column of the piece to move. \\
    \texttt{from\_y}
        & \texttt{integer}
        & The row of the piece to move. \\
    \texttt{to\_x}
        & \texttt{integer}
        & The destination column. \\
    \texttt{to\_y}
        & \texttt{integer}
        & The destination row. \\
\end{CommandParameters}

\subsubsection{Response parameters}

\begin{ResponseParameters}
    \texttt{can\_move}
        & \texttt{boolean}
        & Wether the provided move is valid or not. \\
\end{ResponseParameters}

\subsubsection{Response errors}

\begin{ResponseErrors}
    \texttt{invalid\_coordinates}
        & Provided coordinates are outside the board. \\
    \texttt{no\_piece\_at\_coordinates}
        & There is no piece on the board at the provided coordinates. \\
\end{ResponseErrors}

\newpage
\subsection{move}

\subsubsection{Command parameters}

\begin{CommandParameters}
    \texttt{from\_x}
        & \texttt{integer}
        & The column of the piece to move. \\
    \texttt{from\_y}
        & \texttt{integer}
        & The row of the piece to move. \\
    \texttt{to\_x}
        & \texttt{integer}
        & The destination column. \\
    \texttt{to\_y}
        & \texttt{integer}
        & The destination row. \\
    \texttt{lenient}
        & \texttt{boolean?}
        & If \texttt{true}, skips checking if the move is valid (default: \texttt{false}). \\
\end{CommandParameters}

\subsubsection{Response parameters}

\begin{ResponseParameters}
    \texttt{can\_move}
        & \texttt{boolean}
        & Wether the provided move is valid or not. \\
\end{ResponseParameters}

\subsubsection{Response errors}

\begin{ResponseErrors}
    \texttt{invalid\_move}
        & Provided move is invalid. \\
\end{ResponseErrors}

\newpage
\subsection{exit}

\subsubsection{Command parameters}

n/a

\subsubsection{Response parameters}

n/a

\subsubsection{Response errors}

n/a

