\subsection{unserialize}

\subsubsection{Command parameters}

\begin{CommandParameters}
    \texttt{format}
        & \texttt{enum \{ PGN, binary-64 \}}
        & Either \texttt{PGN} or \texttt{binary-64} (\hyperref[serialization]{§ \ref{serialization}}). \\
    \texttt{value}
        & \texttt{string}
        & The value to deserialize using the specified format. \\
    \texttt{lenient}
        & \texttt{boolean?}
        & If \texttt{true}, skips checking moves are valid (default: \texttt{false}). \\
\end{CommandParameters}

\subsubsection{Response parameters}

n/a

\subsubsection{Response errors}

\begin{ResponseErrors}
    \texttt{invalid\_format}
        & Provided value could not be parsed. \\
    \texttt{invalid\_data}
        & Input data is incorrect. \\
    \texttt{invalid\_move}
        & Input data contains an invalid move. \\
\end{ResponseErrors}
