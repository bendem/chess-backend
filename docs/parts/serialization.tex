\section{Serialization}
\label{serialization}

\texttt{chess\_backend} uses 2 main formats for serialization: \href{http://www.saremba.de/chessgml/standards/pgn/pgn-complete.htm}{PGN} (Portable Game Notation) and the custom binary-64 format.

\subsection{binary-64 serialization}

The \texttt{binary-64} game format is a base64 encoded format separated in two parts.
The header contains a serie of flags giving information about the state of the game:

\begin{center}
\begin{tabu} to\linewidth {X[c, -1] | X[l, 2]}
    Bit & Description \\ \hline\hline
    1 & Did the white king move? \\
    2 & Did the black king move? \\
    3 & Did the white king get checked? \\
    4 & Did the black king get checked? \\
    5 & Is it the whites turn? \\
\end{tabu}
\end{center}
%
The rest of the format represents the pieces on the board in any order encoded using chunks of 7 bits:

\begin{center}
\begin{tabu} to\linewidth {X[c, -1] | X[l, 2]}
    Number of bits & Description \\ \hline \hline
    1 & Color (\texttt{0}: white, \texttt{1}: black). \\
    3 & Row of the piece. \\
    3 & Column of the piece. \\
\end{tabu}
\end{center}
%
\emph{Note that the binary-64 format doesn't provide any validation since you only provide a single moment of the game and not all the moves taken to get there. As such, the rule stating that "the same board layout can only appear 3 times in the same game" cannot be checked.}
